\documentclass[12pt]{amsart}
\usepackage{amsmath}
\usepackage{amsthm}
\usepackage{amsfonts}
\usepackage{amssymb}
\usepackage[margin=1in]{geometry}
\usepackage{hyperref}
\hypersetup{
    colorlinks=true,
    linkcolor=blue
}

\theoremstyle{definition}
\newtheorem{theorem}{Theorem}[section]
\newtheorem{lemma}[theorem]{Lemma}
\newtheorem{definition}[theorem]{Definition}
\newtheorem{corollary}[theorem]{Corollary}
\newtheorem{proposition}[theorem]{Proposition}
\newtheorem{conjecture}[theorem]{Conjecture}
\newtheorem{remark}[theorem]{Remark}
\newtheorem{example}[theorem]{Example}
\newtheorem{problem}[theorem]{Problem}
\newtheorem{notation}[theorem]{Notation}
\newtheorem{question}[theorem]{Question}
\newtheorem{caution}[theorem]{Caution}

\begin{document}

\title{Homework 1 Answers}

\maketitle

For this week, please answer the following questions from the text. 
I've copied the problem itself below and the question numbers for 
your convenience. 

\begin{enumerate}
	\item (1.2) Decrypted Caesar Cyphers, obtained using a brute-force Java program and looking through all 25 possible shifts
		\begin{itemize}
			\item \texttt{ithinkthatishallneverseeabillboardlovelyasatree (With Shift 23)}
			\item \texttt{loveisnotlovewhichalterswhenitalterationfinds (With Shift 17)}
			\item \texttt{inbaitingamousetrapwithcheesealwaysleaveroomforthemouse (With Shift 7)}
		\end{itemize}
	\item (1.3) Use the simple substitution table below
	\begin{center}
		\begin{tabular}{|c |c |c |c |c |c |c |c |c |c |c |c |c |c |c |c |c |c |c |c |c |c |c |c| c| c|}
			\hline
			a & b & c & d & e & f & g & h & i & j & k & l & m & n & o & 
			p & q & r & s & t & u & v & w & x & y & z \\
			\hline
			S & C & J & A & X & U & F & B & Q & K & T & P & R & W & E & 
			Z & H & V & L & I & G & Y & D & N & M & O \\
			\hline
		\end{tabular}
	\end{center}
	\begin{enumerate}
		\item Encrypt the plaintext message
		\begin{center}
			\texttt{The gold is hidden in the garden.}
		\end{center}
		\item Make a decryption table, that is, make a table in which the ciphertext 
			alphabet is in order from A to Z and the plaintext alphabet is mixed up.
		\item Use your decryption table from (b) to decrypt the following message.
		\begin{center}
			\texttt{IBXLX JVXIZ SLLDE VAQLL DEVAU QLB}
		\end{center}
	\end{enumerate}
\item (1.4.c) 

	“A Brilliant Detective”
	\begin{center}
		\ttfamily
		iamfa irlyf amili arwit hallf ormso fsecr etwri tinga ndamm yself
autho rofat rifli ngmon ograp hupon subje ctinw hichi analy zeone
hundr edsep arate ciphe rsbut iconf essth atthi sisen tirel ynewt
omeob jecto fthos ewhoi nvent edthi ssyst emhas appar ently beent
oconc ealth atthe secha racte rscon veyam essag eandt ogive ideat
hatth eyare merer andom sketc hesof child ren

"I am fairly familiar with all forms of secret writing and 
am myself author of a trifling monograph upon [the] subject 
in which I analyze one hundred separate ciphers, but I confess 
that this is entirely new to me. 
[The] object of those who invented this system has 
apparently been to conceal that these characters convey a message and to give 
[the] idea that they are mere random sketches of children."


Process: V seemed obvious to be 'e'. XCSXXC appears 3 times, using the letter frequencies the most likely deciphering was 'thatth'. 
From there G was likely to be 'i' to make 'this' a few times, and Z to be 'm' to make “I am”.  B -> 'y' and N -> 'r' creates the phrase 
'they are'. U -> 'l' creates some adverbs and other familiar combinations, E -> 'f' makes almost the entire first line make sense. 
From there it was trivial to replace the few remaining letters with the only things that make sense. 

	\end{center}
		
		
	\item (1.5) Suppose that you have an alphabet of 26 letters.
		\begin{enumerate}
			\item  How many possible simple substitution ciphers are there?
			\item  A letter in the alphabet is said to be fixed if the encryption of the letter is the
				letter itself. How many simple substitution ciphers are there that leave:
			\begin{enumerate}
				\item No letters fixed?
				\item At least one letter fixed?
				\item Exactly one letter fixed?
				\item At least two letters fixed?
			\end{enumerate}
			(Part (b) is quite challenging! You might try doing the problem first with an alphabet 
			of four or five letters to get an idea of what is going on.)
		\end{enumerate}
\end{enumerate}

\end{document}